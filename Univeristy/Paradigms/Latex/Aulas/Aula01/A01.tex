% -------------------------------------------------
% Prof. Dr. Ausberto S. Castro Vera
% UENF - CCT - LCMAT - Curso de Ci\^{e}ncia da Computa\c{c}\~{a}o - Matem\'{a}tica - PROFMAT
% Campos, RJ,  2013-2024
% Curso de LaTeX 
% -------------------------------------------------


\documentclass{article}    % Tipo de documento ARTIGO

%
% Preambulo
%

\begin{document}

Mudan\c{c}a de par\'{a}grafo SOMENTE com uma linha em BRANCO ...




Paragrafo 01
Corpo-do-Documento Corpo-do-Documento Corpo-do-Documento 123456789 Corpo-do-Documento Corpo-do-Documento Corpo-do-Documento
Corpo-do-Documento Corpo-do-Documento Corpo-do-Documento
Corpo-do-Documento Corpo-do-Documento Corpo-do-Documento
Corpo-do-Documento Corpo-do-Documento Corpo-do-Documento
Corpo-do-Documento Corpo-do-Documento Corpo-do-Documento
Corpo-do-Documento Corpo-do-Documento Corpo-do-Documento
Corpo-do-Documento Corpo-do-Documento Corpo-do-Documento
Corpo-do-Documento Corpo-do-Documento Corpo-do-Documento\\
(tirado da Internet) A \'{a}rea do tri\^{a}ngulo pode ser calculada atrav\'{e}s das medidas da base e da altura da figura. Lembre-se que o tri\^{a}ngulo \'{e} uma figura geom\'{e}trica plana formada por tr\^{e}s lados. Contudo, h\'{a} diversas... A geometria plana ou euclidiana \'{e} a parte da matem\'{a}tica que estuda as figuras que n\~{a}o possuem volume.




Paragrafo 02
Os programas para computadores hoje s\~{a}o bastante complexos. Assim, sempre h\'{a} diversos modos de executar a mesma tarefa em um computador bem como podem existir diferen\c{c}as marcantes entre dois ou mais computadores, em fun\c{c}\~{a}o das diferen\c{c}as em termos de equipamento (hardware), programas usados (software) e modos de configura\c{c}\~{a}o, tanto do equipamento como dos programas.
Na minha experi\^{e}ncia, o aprendizado de recursos de inform\'{a}tica e uso de computadores \'{e} pouco produtivo se feito passo-a-passo, com aulas e mais aulas explanat\'{o}rias. Tenho preferido um m\'{\i}nimo de introdu\c{c}\~{a}o, permitindo bastante trabalho pr\'{a}tico e muita intera\c{c}\~{a}o entre as pessoas. Assim, al\'{e}m da ajuda intensiva do professor, que uns precisam e outros n\~{a}o, quem sabe mais ensina pouco a pouco quem sabe menos.\\





Paragrafo 03
Corpo-do-Documento
Corpo-do-Documento
Corpo-do-Documento
Corpo-do-Documento
Corpo-do-Documento
Corpo-do-Documento
PalavraMuitoCompridaCorpo-do-Documento123456789123456789
A Internet nada mais \'{e} que uma rede de computadores ligados fisicamente e que trocam informa\c{c}\~{a}o de acordo com conven\c{c}\~{o}es pr\'{e}-estabelecidas. Estas conven\c{c}\~{o}es regulam o tr\'{a}fego de informa\c{c}\~{a}o, que pode ser de v\'{a}rios tipos: destinada a correio eletr\^{o}nico, transfer\^{e}ncia de dados e programas e arquivos para visualiza\c{c}\~{a}o imediata. Nesta \'{u}ltima categoria \'{e} que se enquadra a World Wide Web (WWW) ou "teia ao redor do mundo".


\end{document}

