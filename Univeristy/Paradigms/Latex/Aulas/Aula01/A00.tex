% -------------------------------------------------
% Prof. Dr. Ausberto S. Castro Vera
% UENF - CCT - LCMAT - Curso de Ci\^{e}ncia da Computa\c{c}\~{a}o - Matem\'{a}tica - PROFMAT
% Campos, RJ,  2013-2024
% Curso de LaTeX 
% -------------------------------------------------


\documentclass{article}  % Tipo de documento ARTIGO

%
% Preambulo: comandos definidos pelo usu\'{a}rio e uso de pacotes \usepackage{NomePacote}
%
% Qualquer linha de COMENTARIO come\c{c}a com o s\'{\i}mbolo de porcentagem %
%
%%--------------------------------------------------------------------------------------------
\begin{document} %%% somente a partir deste ponto o texto que segue ser\'{a} compilado

Paragrafo 01
Computational thinking (CT) refers to the thought processes involved in formulating problems so their solutions can be represented as computational steps and algorithms. In education, CT is a set of problem-solving methods that involve expressing problems and their solutions in ways that a computer could also execute. It involves automation of processes, but also using computing to explore, analyze, and understand processes (natural and artificial)
 
 
 
 
Paragrafo 02
A educa\c{c}\~{a}o matem\'{a}tica, tamb\'{e}m chamada de did\'{a}tica matem\'{a}tica em pa\'{\i}ses europeus, \'{e} uma \'{a}rea das ci\^{e}ncias sociais que se dedica ao estudo da aprendizagem e ensino da matem\'{a}tica. Est\'{a} na fronteira entre matem\'{a}tica, pedagogia e psicologia (wikipedia).





Paragrafo 03
O ensino e a aprendizagem de Matem\'{a}tica constituem um desafio em todos os
n\'{\i}veis de escolaridade. Muitas vezes, os adolescentes ingressam em uma universidade e
n\~{a}o possuem uma base de conhecimentos matem\'{a}ticos fortalecida. Por outro lado,
outros fatores tamb\'{e}m podem influenciar e aumentar as dificuldades encontradas nesta
disciplina e \'{e} papel da institui\c{c}\~{a}o e do professor promover maneiras de incentivar,
inserir estrat\'{e}gias metodol\'{o}gicas e possibilitar que esse aluno obtenha o conhecimento
necess\'{a}rio para garantir uma forma\c{c}\~{a}o s\'{o}lida no curso superior escolhido (Adriana Borges de Paiva et.al. 2020)


\end{document} %% At\'{e} aqui ser\'{a} compilado qualquer texto
%%--------------------------------------------------------------------------------------------

Qualquer texto a partir deste ponto N\~{A}O ser\'{a} compilado!!!
n\~{a}o
n\~{a}o
n\~{a}o
n\~{a}o ser\'{a} compilado!!!
n\~{a}o
n\~{a}o
n\~{a}o ser\'{a} compilado!!!


