% Prof. Dr. Ausberto S. Castro Vera
% UENF - CCT - LCMAT - Curso de Ci\^{e}ncia da Computa\c{c}\~{a}o
% Campos, RJ,  2023
% Disciplina: Paradigmas de Linguagens de Programa\c{c}\~{a}o
% Aluno:



\chapter{ Introdu\c{c}\~{a}o}

Python \'{e} uma poderosa linguagem de programa\c{c}\~{a}o de alto n\'{\i}vel e orientada a objetos, originalmente conceitualizada por Guido van Rossum, no final dos anos 1980, no National Research Institute of Mathematics and Computer Science, Holanda. Python foi a sucessora da linguagem ABC. Python \'{e} uma linguagem  de uso geral, orientada a objetos, com c\'{o}digo bastante leg\'{\i}vel, e com muitas bibliotecas dispon\'{\i}veis e amplamente conhecidas (NumPy, SciPy, Pandas, IPython, Matplotlib, mIPy, ScraPy, etc.)
\begin{quote}
  Python, uma linguagem de script de c\'{o}digo aberto, se tornou a linguagem de ensino introdut\'{o}ria mais popular nas principais universidades americanas - entre elas, Georgia Tech - segundo uma pesquisa recente de Philip Guo, professor assistente de ci\^{e}ncia da computa\c{c}\~{a}o na Universidade de Rochester. Guo decidiu conduzir a pesquisa depois de notar, nos \'{u}ltimos anos, que o Python estava substituindo linguagens como Java como a introdu\c{c}\~{a}o de fato \`{a} classe de programa\c{c}\~{a}o em mais e mais aulas de ci\^{e}ncia da computa\c{c}\~{a}o em universidades de todo o pa\'{\i}s. \cite{Shein2015}
\end{quote}


   \section{Aspectos hist\'{o}ricos da linguagem FORTRAN}

A hist\'{o}ria da maioria de linguagens de programa\c{c}\~{a}o n\~{a}o tem uma data fixa, nem um autor \'{u}nico. A sua evolu\c{c}\~{a}o inclui muitos personagens, muitas institui\c{c}\~{o}es e muitas vers\~{o}es.

A seguir, menciona-se alguns aspectos hist\'{o}ricos da linguagem FORTRAN, baseados em \cite{Backus1998}, \cite{Adams1996} :
\begin{itemize}
  \item
  \item
\end{itemize}


   \section{\'{A}reas de Aplica\c{c}\~{a}o da Linguagem}
   Esta linguagem \'{e} utilizada e aplicada nas seguintes \'{a}reas: !!!!! As aqui mostradas s\~{a}o exemplos!!!

        \subsection{ Supercomputa\c{c}\~{a}o}
        Fazer uma breve descri\c{c}\~{a}o. Pelo menos 3 par\'{a}grafos mencionando exemplos

        \subsection{ Programa\c{c}\~{a}o Cient\'{\i}fica}
        Fazer uma breve descri\c{c}\~{a}o. Pelo menos 3 par\'{a}grafos mencionando exemplos

        \subsection{ outras} 