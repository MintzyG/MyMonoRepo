% Prof. Dr. Ausberto S. Castro Vera
% UENF - CCT - LCMAT - Curso de Ci\^{e}ncia da Computa\c{c}\~{a}o
% Campos, RJ,  2023
% Disciplina: Paradigmas de Linguagens de Programa\c{c}\~{a}o
% Aluno:



\chapter{ Aplica\c{c}\~{o}es da Linguagem Fortran}

Ver uma lista de aplica\c{c}\~{o}es Fortran aqui:
\url{https://people.sc.fsu.edu/~jburkardt/f_src/f_src.html}

Devem ser mostradas pelo menos CINCO aplica\c{c}\~{o}es completas da linguagem, e em cada caso deve ser apresentado:
\begin{itemize}
  \item Uma breve descri\c{c}\~{a}o da aplica\c{c}\~{a}o
  \item O c\'{o}digo completo da aplica\c{c}\~{a}o,
  \item Imagens do c\'{o}digo fonte no compilador,
  \item Imagens dos resultados ap\'{o}s a compila\c{c}\~{a}o-interpreta\c{c}\~{a}o do c\'{o}digo fonte
  \item Links e referencias bibliogr\'{a}ficas de onde foi obtido a aplica\c{c}\~{a}o
\end{itemize}



    %%%--------------------------------------------------------------------
    \section{Opera\c{c}\~{o}es b\'{a}sicas}
    %%%--------------------------------------------------------------------
    Implementar um Programa INTERATIVO para calcular o VOLUME de um cilindro (menu interativo)
    \begin{itemize}
      \item \textbf{Descri\c{c}\~{a}o da aplica\c{c}\~{a}o}:
      \item \textbf{C\'{o}digo Fortran completo da aplica\c{c}\~{a}o}:
      \item \textbf{Capturas de tela da aplica\c{c}\~{a}o rodando no compilador}:
      \item \textbf{Capturas de telas dos RESULTADOS da aplica\c{c}\~{a}o}:
      \item \textbf{Refer\^{e}ncias}: bibliografia, links da Internet, etc.
    \end{itemize}



    %%%--------------------------------------------------------------------
    \section{O algoritmo Quicksort en Fortran}
    %%%--------------------------------------------------------------------
    Este algoritmo esta dispon\'{\i}vel na internet: s\'{o} copiar, adaptar, comentar o c\'{o}digo e compilar
    \begin{itemize}
      \item \textbf{Descri\c{c}\~{a}o da aplica\c{c}\~{a}o}:
      \item \textbf{C\'{o}digo Fortran completo da aplica\c{c}\~{a}o}:
      \item \textbf{Capturas de tela da aplica\c{c}\~{a}o rodando no compilador}:
      \item \textbf{Capturas de telas dos RESULTADOS da aplica\c{c}\~{a}o}:
      \item \textbf{Refer\^{e}ncias}: bibliografia, links da Internet, etc.
    \end{itemize}


    %%%--------------------------------------------------------------------
    \section{Programa de C\'{a}lculo Num\'{e}rico}
    %%%--------------------------------------------------------------------
    Implementar um programa COMPLETO. Pode ser a solu\c{c}\~{a}o de um sistema de equa\c{c}\~{o}es, o c\'{a}lculo das ra\'{\i}zes de uma fun\c{c}\~{a}o,  interpola\c{c}\~{a}o, etc.
    \begin{itemize}
      \item \textbf{Descri\c{c}\~{a}o da aplica\c{c}\~{a}o}:
      \item \textbf{C\'{o}digo Fortran completo da aplica\c{c}\~{a}o}:
      \item \textbf{Capturas de tela da aplica\c{c}\~{a}o rodando no compilador}:
      \item \textbf{Capturas de telas dos RESULTADOS da aplica\c{c}\~{a}o}:
      \item \textbf{Refer\^{e}ncias}: bibliografia, links da Internet, etc.
    \end{itemize}


    %%%--------------------------------------------------------------------
    \section{Aplica\c{c}\~{a}o usando Matrizes}
    %%%--------------------------------------------------------------------
    \begin{itemize}
      \item \textbf{Descri\c{c}\~{a}o da aplica\c{c}\~{a}o}:
      \item \textbf{C\'{o}digo Fortran completo da aplica\c{c}\~{a}o}:
      \item \textbf{Capturas de tela da aplica\c{c}\~{a}o rodando no compilador}:
      \item \textbf{Capturas de telas dos RESULTADOS da aplica\c{c}\~{a}o}:
      \item \textbf{Refer\^{e}ncias}: bibliografia, links da Internet, etc.
    \end{itemize}

    %%%--------------------------------------------------------------------
    \section{Aplica\c{c}\~{o}es Profissionais}
    %%%--------------------------------------------------------------------
    Aqui pode ser qualquer aplica\c{c}\~{a}o de outra \'{a}rea de conhecimento, por exemplo: F\'{\i}sica, Mec\~{a}nica de Flu\'{\i}dos, Biologia, Astronomia, Jogos, Qu\'{\i}mica, etc. Pesquisar na Internet, para aplica\c{c}\~{o}es prontas e pequenas.
    \begin{itemize}
      \item \textbf{Descri\c{c}\~{a}o da aplica\c{c}\~{a}o}:
      \item \textbf{C\'{o}digo Fortran completo da aplica\c{c}\~{a}o}:
      \item \textbf{Capturas de tela da aplica\c{c}\~{a}o rodando no compilador}:
      \item \textbf{Capturas de telas dos RESULTADOS da aplica\c{c}\~{a}o}:
      \item \textbf{Refer\^{e}ncias}: bibliografia, links da Internet, etc.
    \end{itemize}

