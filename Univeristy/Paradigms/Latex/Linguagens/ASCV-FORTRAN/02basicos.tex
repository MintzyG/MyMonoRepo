% Prof. Dr. Ausberto S. Castro Vera
% UENF - CCT - LCMAT - Curso de Ci\^{e}ncia da Computa\c{c}\~{a}o
% Campos, RJ,  2023
% Disciplina: Paradigmas de Linguagens de Programa\c{c}\~{a}o
% Aluno:



\chapter{ Conceitos b\'{a}sicos da Linguagem FORTRAN}

Os livros b\'{a}sicos para o estudo da Linguagem PORTRAN s\~{a}o: \cite{Chivers2018}, \cite{Sebesta2018}, \cite{Metcalf2018}, \cite{Chapman2018}

Neste cap\'{\i}tulo \'{e} apresentado ....  Segundo \cite{Sebesta2018}, a linguagem FORTRAN,  . . .

De acordo com \cite{Sebesta2018} e \cite{roy04}, a linguagem Python . . . \cite{Sebesta2018} afirma que a linguagem Python . . .

Considerando que a linguagem FORTRAN (\cite{Sebesta2018}, \cite{Watt1990}) \'{e} considerada como ....

    %%%%%%%%=================================
    \section{Vari\'{a}veis e constantes}
    %%%%%%%%=================================


    %%%%%%%%=================================
    \section{Tipos de Dados B\'{a}sicos}
    %%%%%%%%=================================

     %%%........................
            \subsection{String}
     %%%........................

    C\'{o}digo fonte para a linguagem FORTRAN:
    \begin{lstlisting}
    PROGRAM Triangulo
     IMPLICIT NONE
     REAL :: a, b, c, Area
     PRINT *, 'Bemvindo. Por favor ingresse o&
              &comprimento dos 3 lados.'
     READ *, a, b, c
     PRINT *, 'Area do Triangulo:  ', Area(a,b,c)
    END PROGRAM Triangulo

    FUNCTION Area(x,y,z)
     IMPLICIT NONE
     REAL :: Area            ! tipo de funcao
     REAL, INTENT( IN ) :: x, y, z
     REAL :: theta, height
     theta = ACOS((x**2+y**2-z**2)/(2.0*x*y))
     height = x*SIN(theta); Area = 0.5*y*height
    END FUNCTION Area
    \end{lstlisting}


    %%%%%%%%=================================
    \section{Operadores e Express\~{o}es em Fortran}
    %%%%%%%%=================================


