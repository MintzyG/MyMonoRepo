% Prof. Dr. Ausberto S. Castro Vera
% UENF - CCT - LCMAT - Curso de Ci\^{e}ncia da Computa\c{c}\~{a}o
% Campos, RJ,  2023
% Disciplina: Paradigmas de Linguagens de Programa\c{c}\~{a}o
% Aluno: Eric Hoffmann Fernandes Braga


\chapterimage{ScalaH} % Chapter heading image ==>  Trocar este arquivo por outro 1200x468
\chapter{ Conceitos b\'{a}sicos da Linguagem Scala}

Os livros b\'{a}sicos para o estudo da Linguagem Scala s\~{a}o: \cite{Whaling2020}, \cite{Wampler2021}, \cite{Hunt2018} e \cite{Upadhyaya2019}

Neste cap\'{\i}tulo \'{e} apresentado ....  Segundo \cite{Hunt2018}, a linguagem Scala,  . . .

De acordo com \cite{Sebesta2018} e \cite{roy04}, a linguagem Scala . . . \cite{Sebesta2018} afirma que a linguagem Python . . .

Considerando que a linguagem Scala (\cite{Whaling2020}, \cite{Upadhyaya2019}) \'{e} considerada como ....

    %%%%%%%%=================================
    \section{Vari\'{a}veis e constantes}
    %%%%%%%%=================================


    %%%%%%%%=================================
    \section{Tipos de Dados B\'{a}sicos}
    %%%%%%%%=================================

     %%%........................
            \subsection{String}
     %%%........................

    C\'{o}digo fonte para a linguagem Scala:
    \begin{lstlisting}
    object HelloWorld {
  def main(args: Array[String]): Unit = {
    println("Hello, world!")
  }
}
    \end{lstlisting}

    \begin{lstlisting}
  class Rational(n: Int, d: Int) {

    require(d != 0)

    private val g = gcd(n.abs, d.abs)
    val numer = n / g
    val denom = d / g

    def this(n: Int) = this(n, 1)

    def + (that: Rational): Rational =
      new Rational(
        numer * that.denom + that.numer * denom,
        denom * that.denom
      )

    def + (i: Int): Rational =
      new Rational(numer + i * denom, denom)

    def - (that: Rational): Rational =
      new Rational(
        numer * that.denom - that.numer * denom,
        denom * that.denom
      )
    \end{lstlisting}
    %%%%%%%%=================================
    \section{Operadores e Express\~{o}es em Scala}
    %%%%%%%%=================================


