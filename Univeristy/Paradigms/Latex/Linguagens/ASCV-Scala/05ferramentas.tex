% Prof. Dr. Ausberto S. Castro Vera
% UENF - CCT - LCMAT - Curso de Ci\^{e}ncia da Computa\c{c}\~{a}o
% Campos, RJ,  2023
% Disciplina: Paradigmas de Linguagens de Programa\c{c}\~{a}o
% Aluno:



\chapterimage{ScalaH} % Chapter heading image ==>  Trocar este arquivo por outro 1200x468
\chapter{Ferramentas existentes e utilizadas}

Neste cap\'{\i}tulo devem ser apresentadas pelo menos DUAS (e no m\'{a}ximo 5) ferramentas consultadas e utilizadas para realizar o trabalho, e usar nas aplica\c{c}\~{o}es. Considere em cada caso:
\begin{itemize}
  \item Nome da ferramenta (compilador-interpretador)
  \item Endere\c{c}o na Internet
  \item Vers\~{a}o atual e utilizada
  \item Descri\c{c}\~{a}o simples (m\'{a}x 2 par\'{a}grafos)
  \item Telas capturadas da ferramenta
  \item Outras informa\c{c}\~{o}es
\end{itemize}


    \section{Editores para Fortran}


    \section{Compiladores}
            \begin{itemize}
              \item Site principal : \url{https://www.scala-lang.org/}
              \item Scala 3 : \url{https://www.scala-lang.org/download/}
              \item
              \item
              \item
            \end{itemize}



    \section{Ambientes de Programa\c{c}\~{a}o IDE para Scala}
